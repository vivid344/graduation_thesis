% 公立はこだて未来大学 卒業論文 テンプレート ver1.50
% (c) Junichi Akita (akita@fun.ac.jp), 2003.10.31
% update by N.T.,  2004.11.10
%
\documentclass{funthesis}
%¥documentclass[english]{funthesis} % use [english] option for English style

\usepackage{graphicx} % 図(EPS形式)を本文中で読み込む場合はこれを宣言
\usepackage{url}

% この部分に,タイトル・氏名などを書く.
% タイトルなどの定義の始まり
\jtitle{知識ベース型推薦を用いたフードツーリズム支援システムの構築}% 論文の和文タイトル
%
\etitle{Development of a Food Tourism Support System Using Knowledge-based Recommendation}% 論文の英文タイトル
%
\htitle{Development of a Food Tourism Support System Using Knowledge-based Recommendation}   % ヘッダー用の論文の短縮英文タイトル
%     必ず1行に収まるように英文タイトルを短縮する.
%
\jauthor{三好 良弥}     % 氏名(日本語)
\eauthor{Ryoya Miyoshi}   % 氏名(英語)
\jaffiliciation{情報アーキテクチャ学科} % 所属学科名(日本語)
\eaffiliciation{Department of Media Architecture} % 所属学科名(英語)
\studentnumber{1014127}   % 学籍番号
\jadvisor{奥野 拓}    % 正指導教員名(日本語)
%¥jcoadvisor{} % 副指導教員(日本語)がいる場合は
                        % コメントアウトし名前を書く
                        % 副指導教員がいない場合は,ここは削除しても可
\eadvisor{Taku OKUNO}  % 正指導教員名(英語)
%¥ecoadvisor{}   % 副指導教員(英語)がいる場合は
                         % コメントアウトし名前を書く
                         % 副指導教員がいない場合は,ここは削除しても可
\jdate{平成30年01月31日}    % 論文提出日   (日本語)
\edate{January 31, 2018}     % 論文提出年月 (英語)
% タイトルなどの定義の終わり

\begin{document}

%--------------------------------------------------------------------
\maketitle       % タイトルページを作成

%--------------------------------------------------------------------
% 英文概要(250語程度)
\begin{eabstract}
hoge hoge hoge hoge hoge hoge hoge hoge hoge hoge hoge hoge hoge hoge hoge hoge hoge hoge hoge hoge hoge hoge hoge hoge hoge hoge hoge hoge hoge hoge hoge hoge hoge hoge hoge hoge hoge hoge hoge hoge hoge hoge hoge hoge hoge hoge hoge hoge hoge hoge hoge hoge hoge hoge hoge hoge hoge hoge hoge hoge hoge hoge hoge hoge hoge hoge hoge hoge hoge hoge hoge hoge hoge hoge hoge hoge hoge hoge hoge hoge hoge hoge hoge hoge hoge hoge hoge.
\end{eabstract}

% 英文キーワード(5個程度をコンマ(,)で区切って羅列する)
\begin{ekeyword}
Kowledge-based Recommender System
\end{ekeyword}

%--------------------------------------------------------------------
% 和文概要(300字程度)
\begin{jabstract}
近年,地域らしい料理を食べることを目的とした旅であるフードツーリズムが盛んである.しかし,観光客の嗜好によって地域らしい料理の判断基準が異なるため,従来のグルメサイトでは地域らしい料理を探すことが困難である.この問題を解決するために,観光客の嗜好を考慮した地域らしい料理推薦システムを提案する.グルメサイトからWebスクレイピングを行い抽出した飲食店及び料理の情報を用いてデータベースを構築し,観光客の嗜好にあった地域らしい料理の推薦を行う.推薦手法として,「評価が高い商品」や「1000円以下の商品」などユーザが商品に求める具体的な条件がある場合に有効な知識ベース型推薦を用いることで,嗜好にあった料理の推薦を可能にする.

\end{jabstract}

% 和文キーワード(5個程度をコンマ(,)で区切って羅列する)
\begin{jkeyword}
知識ベース型推薦
\end{jkeyword}

%--------------------------------------------------------------------
\tableofcontents % 目次を作成



%--------------------------------------------------------------------
% ¥includegraphics[width=??cm]{hoge.eps} % 図(EPS形式)を読み込む場合


%--------------------------------------------------------------------
\chapter{序論} 
\section{背景}
近年,ニューツーリズムの振興[1] により,地域らしい料理が重要な観光資源となっている.こうした,地域らしい料理を食べることを目的とした旅のことをフードツーリズムと呼ぶ.日本フードツーリズム協会[2] は,フードツーリズムを,地域ならではの料理・食文化をその地域で楽しむための旅と定義している.また,フードツーリズムには地域らしい料理・食文化,それを引き立てる体験,場所,人の4 つの要素があると定義している.

じゃらんが実施したアンケート[3] では,観光客が観光地を選んだ理由のひとつとして「地域らしい料理・特産品に興味があったから」と回答した人が41.6\%であった.このことから,観光客は地域らしい料理を旅行の際に重要視していることがわかる.

観光客がフードツーリズムの重要な要素である地域らしい料理を探す方法の一つとしてグルメサイトを用いて検索する方法がある.しかし,観光客によって地域らしい料理に求める条件が異なるため,観光客が期待する料理を探すことは容易ではない.

\section{研究目的}  
本研究では,観光客の嗜好や状況を考慮した地域らしい料理の推薦を行うことで観光客の満足度向上を目指す.そこで,本研究では知識ベース型推薦を用いた地域らしい料理推薦システムを提案する.

\section{本論文の構成}
本論文は全6章で構成されている.第1章では,本研究の背景および研究目的について述べた.第2章では,関連技術について述べる.第3章では,関連研究について述べる.第4章では,本研究で提案する地域らしい料理推薦システムの構築について述べる.第5章では,実験方法および実験の結果,結果から得られた考察について述べる.最後に第6章では,本研究のまとめと今後の展望を述べる.

%--------------------------------------------------------------------
\chapter{関連技術}

%--------------------------------------------------------------------
\chapter{関連研究}

%--------------------------------------------------------------------
\chapter{提案手法}

%--------------------------------------------------------------------
\chapter{実験}

%--------------------------------------------------------------------
\chapter{結論}
\section{まとめ}
\section{今後の展望}
\chapter *{謝辞}



%--------------------------------------------------------------------
% 参考文献
\begin{thebibliography}{7}
\end{thebibliography}

% 表目次の表示
\listoftables

% 図目次の表示
\listoffigures

\end{document}
